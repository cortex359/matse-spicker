\section{Formale Sprachen}
\subsection{Backus-Naur-Form}

\begin{defi}{(kontextfreie) Grammatik}
    Eine (kontextfreie) Grammatik $G$ ist ein 4-Tupel $\{N, T, \Sigma, P\}$ mit folgenden Eigenschaften:
    \begin{itemize}
        \item $N$ ist eine endliche Menge von \emph{Nichtterminalsymbolen},
        \item $T$ ist eine endliche Menge von \emph{Terminalsymbolen},
        \item ein \emph{Startsymbol} $\Sigma \in N$,
        \item eine endliche Menge an Produktionsregeln $P \subset N \times T^*$.
    \end{itemize}
    Es gilt $N \cap M = \emptyset$. $^*$ bezeichnet die Kleensche Hülle.

    Anmerkung: Die Notation verhält sich in der Vorlesung sehr anders als in den meisten Quellen.
\end{defi}

\begin{defi}{Backus-Naur-Form}
    In der \emph{Backus-Naur-Form} werden Symbole durch eine Zuweisung definiert.\\
    Diese wird durch den Operator \texttt{:=} dargestellt.

    \begin{itemize}
        \item Dabei gilt weiterhin, dass in der Zuweisung eines \emph{Nichtterminalsymbols} mindestens ein weiteres Symbol auftauchen muss und
        \item bei der Zuweisung eines \emph{Terminalsymbols} nur Zeichen des Alphabets.
        \item Bei einer Zuweisung können mehrere Symbole oder Zeichen(-ketten) des Alphabets verbunden werden.
        \item Durch ein Leerzeichen (\glqq ~~~\grqq, \texttt{space}) wird eine \texttt{und}-Verknüpfung definiert,
        \item durch einen senkrechten Stricht (\glqq ~$\mid$~\grqq, \texttt{pipe}) wird eine \texttt{oder}-Zuweisung definiert.
        \item Zur logischen Gliederung der Verknüpfungen können Klammern verwendet werden.
        \item Es ist ebenfalls möglich Kardinalitäten für Symbole zu definieren:
              \subitem Geschweifte Klammern \texttt{\{\}} definieren dabei eine \emph{beliebige Wiederholung} und
              \subitem eckige Klammern \texttt{[]} definieren ein \emph{einmaliges Auftreten}.
              \subitem In beiden Fällen ist das Weglassen des Symbols ebenfalls möglich.
    \end{itemize}
\end{defi}

\begin{example}{Backus-Naur-Form}
    \begin{allintypewriter}
        \begin{tabular}{rl}
            Satz        & := Subjekt Verb Präposition Nomen Ende                       \\
            Subjekt     & := Nomen                                                     \\
            Nomen       & := Artikel Substantiv                                        \\
            Artikel     & := \ditto der\ditto | \ditto die\ditto | \ditto das\ditto    \\
            Substantiv  & := \ditto Mann\ditto | \ditto Frau\ditto | \ditto Haus\ditto \\
            Verb        & := \ditto geht\ditto                                         \\
            Präposition & := \ditto in\ditto                                           \\
            Ende        & := \ditto .\ditto
        \end{tabular}
    \end{allintypewriter}
\end{example}

\subsection{Programmiersprachen}

\begin{defi}{Compiler}
    Das Programm, welches den
    menschenlesbaren \emph{Quellcode} anhand der Regeln der zugrundeliegenden Grammatik
    interpretiert und in maschinenlesbare Anweisungen überführt, wird \emph{Compiler} genannt.

    Die maschinenlesbaren Anweisungen können dabei entweder direkt in eine
    maschinenspezifische Binärfolge umgewandelt werden, oder in einen sogenannten
    \emph{Bytecode}.

    Im zweiten Fall kann der Bytecode,
    auch Zwischencode genannt, noch an beliebigen Prozessortypen angepasst werden,
    muss dafür aber vor der Ausführung erneut bearbeitet werden.

    Bei der Kompilierung eines Programms werden zwei Phasen durchlaufen:
    \begin{enumerate}
        \item \textbf{Analysephase}, in der der Quellcode analysiert und auf Fehler geprüft wird:
              \subitem \underline{\emph{Lexer} (lexikalischer Scanner)}: \\
              Quellcode wird in einzelne Teile (\emph{Tokens}) zerlegt und klassifiziert (z.B. als \emph{Schlüsselwörter} oder \emph{Bezeichner}).\\
              Der Lexer erkennt hier z.B. falsch benannte Variablen.
              \subitem \underline{\emph{Parser} (syntaktische Analyse)}: \\
              Vom Lexer erzeugte \emph{Tokens} werden dem \emph{Parser} übergeben.
              Der Parser überprüft den Quellcode auf Fehler und setzt ihn bei Fehlerfreiheit in einen \emph{Syntaxbaum} (AST) um.\\
              Hier werden Fehler wie fehlende Semikolons oder falsche genutzte Operatoren erkannt.
              \subitem \underline{\emph{Semantische Analyse}}: \\
              Hier wird z.B. kontrolliert, ob eine verwendete Variable auch vorher deklariert wurde, oder ob Quell- und Zieltyp einer Anweisung übereinstimmen. Dabei erzeugte Metadaten werden in den AST integriert.
              In diesem Schritt werden generell semantische Fehler erkannt, d.h. Anweisungen, die korrekt aussehen, aber fehlerhaft sind.
              \\
        \item \textbf{Synthesephase}, in der Binär- bzw. Bytecode erzeugt wird:
              \subitem \underline{\emph{Zwischencodeerzeugung}}: \\
              Hier wird plattformunabhängiger Bytecode erzeugt, der als Grundlage für den nächsten Schritt dient.
              \subitem \underline{\emph{Optimierung}}: \\
              Während der Optimierung wird versucht die Performanz des erzeugten Programms zu steigern.
              \subitem \underline{\emph{Codegenerierung}}: \\
              Hier wird aus dem optimierten Zwischencode der \emph{Binärcode} erzeugt.
    \end{enumerate}
\end{defi}

\begin{bonus}{Ahead-of-Time Compiler}
    Bei dieser Variante wird der gesamte Quellcode \emph{in einem Rutsch}
    in Binär- oder Bytecode übersetzt. Das kann mitunter dazu führen, dass die Kompilierung
    sehr lange dauert.
\end{bonus}

\begin{bonus}{Just-in-Time Compiler}
    Daher haben sich mit der Zeit auch sogenannte \emph{Just-in-
        Time Compiler} etablieren können, die nur einen kleinen Teil des Quellcodes direkt
    übersetzen und weitere Teile des Quellcodes erst dann übersetzen, \emph{wenn sie bei der
        Programmausführung benötigt werden}.
\end{bonus}

\begin{defi}{Linker}
    Nachdem der Compiler den Binärcode erzeugt hat, muss der \emph{Linker} daraus noch
    ein lauffähiges Programm erstellen. Das ist insbesondere der Fall, wenn es mehrere
    Dateien mit Quellcode gibt.

    Dies geschieht indem der Binärcode zu den verschiedenen
    Quellcodedateien, der nun in sogenannten \emph{Objektdateien} vorliegt, zusammengefügt wird. Dabei werden beispielsweise symbolische Adressen aus mehreren
    Quellcodemodulen und externen Bibliotheken so angepasst, dass sie zusammen
    passen.

    Dabei wird zwischen \emph{statischem Linken} und \emph{dynamischen Linken} unterschieden:

    Beim \textbf{statischen Linken} werden alle verfügbaren Objektdateien zu \emph{einer
        einzigen, ausführbaren Programmdatei} gelinkt.
    \begin{itemize}
        \item Vorteil: keine externen Abhängigkeiten, Programm auf jedem geeigneten System ohne Weiteres lauffähig
        \item Nachteil: nicht mehr möglich, einzelne Programmteile auszutauschen ohne den Linkvorgang vollständig zu wiederholen,
    \end{itemize}
    Beim \textbf{dynamischen Linken} werden Funktions- und Variablenadressen erst zur Laufzeit aufgelöst, sodass externe Bibliotheken
    einfach in Form von \emph{Dynamically Linked Libraries} (DLLs) bzw. \emph{Shared Objects} (SOs) angesprochen werden können. Da DLLs bzw. SOs als existierend
    vorausgesetzt werden, wird das fertige Programm bei dynamischem Linken kleiner.
    \begin{itemize}
        \item Vorteil: mehrere Programme können dieselbe externe Bibliothek verwenden, ohne dass der benötigte Code mehrfach in einzelne Programme integriert werden muss.
    \end{itemize}
\end{defi}

\begin{defi}{Interpreter}
    \emph{Interpreter} zeichnen sich dadurch aus, dass der Quellcode nicht einmalig in Binärcode übersetzt wird, sondern bei jeder Programmausführung schrittweise abgearbeitet wird.
    \begin{itemize}
        \item Vorteile: einfache Portabilität, dynamischere Quellcodeverwaltung
        \item Nachteile: deutlich langsamere Ausführungsgeschwindigkeit, keine Optimierungen an Programmstruktur möglich
    \end{itemize}
\end{defi}
