
\section{Datenschutz und Sicherheit}
\subsection{Datensicherheit}

\begin{defi}{Kryptographie}
    Kryptographie ist die Wissenschaft der Verschlüsselung von Informationen. Sie wird genutzt
    um Informationen auf eine Art und Weise zu verändern, sodass ein Unbefugter
    die Informationen nicht mehr lesen kann. Es soll nur dem tatsächlichen Adressaten
    einer Nachricht möglich sein, diese auch zu lesen. Es geht also explizit nicht darum,
    den Diebstahl von Informationen zu verhindern, sondern es geht darum, dass
    entwendete Informationen nicht gelesen werden können.
\end{defi}

\begin{defi}{Transpositionschiffren}
    Die Verschlüsselung wird durch Umordnung der Zeichen
    im Klartext umgesetzt.
\end{defi}

\begin{defi}{Substitutionschiffren}
    Die Verschlüsselung wird durch Ersetzung der Zeichen
    im Klartext realisiert.

    \textbf{Monoalphabetische Substitution}:
    \begin{itemize}
        \item Jedes Zeichen des Klartexts wird durch genau
              ein eindeutiges Zeichen aus dem Geheimtextalphabet ersetzt, das für einen
              gewählten Schlüssel immer gleich ist.
    \end{itemize}

    \textbf{Homophone Substitution}:
    \begin{itemize}
        \item Jedes Zeichen des Klartexts wird durch genau ein
              anderes Zeichen aus einer eindeutigen Menge von Zeichen ersetzt, die für einen
              gewählten Schlüssel immer gleich ist.
    \end{itemize}

    \textbf{Polyalphabetische Substitution}:
    \begin{itemize}
        \item Jedes Zeichen wird durch ein eindeutiges Zeichen
              aus einem von mehreren geheimen Alphabeten ersetzt, die für einen gewählten
              Schlüssel immer gleich sind. Die Alphabete werden dabei immer der Reihe nach
              verwendet.
    \end{itemize}
\end{defi}

\begin{example}{Cäsar Chiffre}
    Im einfachsten Fall beinhalten $\mathfrak{P}$ und $\mathfrak{S}$ dieselben Zeichen und sind nur gegeneinander
    um $k$ Zeichen verschoben. Diese Variante einer \emph{monoalphabetischen Substitutionschiffre} wird \emph{Cäsar Chiffre} genannt.

    Alice wählt für die Verschlüsselung $k = 10$ und erhält damit die folgenden Alphabete:

    \begin{center}
        $\mathfrak{P}$: \texttt{A B C D E F G H I J K L M N O P Q R S T U V W X Y Z}\\
        $\mathfrak{S}$: \texttt{J K L M N O P Q R S T U V W X Y Z A B C D E F G H I}\\
    \end{center}

    Da Alice $k = 10$ gewählt hat, beginnt das Geheimtextalphabet mit \texttt{J}, dem
    zehnten Buchstaben des Alphabets. Die Verschlüsselung führt Alice durch,
    indem sie jeden Buchstaben ihres Klartexts durch den entsprechenden Buchstaben
    aus dem Geheimtextalphabet ersetzt.Werden die beiden Alphabete wie
    weiter oben aufgeschrieben, ist das einfach der Buchstabe im Geheimtextalphabet
    der direkt unter dem Buchstaben im Klartextalphabet steht. Damit ergibt
    sich für die Nachricht \texttt{HALLO} der Geheimtext \texttt{QJUUX}. Bob kann den
    Geheimtext dann entschlüsseln, indem er wiederum beide Alphabete untereinander
    schreibt und jedem Buchstaben aus dem Geheimtext den zugehörigen
    Buchstaben aus dem Klartextalphabet zuorndet, der dann genau darüber steht.
\end{example}

\begin{example}{Vigenère Chiffre}
    Eine bekannte \emph{polyalphabetische Substitutionschiffre} ist die \emph{Vigenère Chiffre}.
    Sie funktioniert im Prinzip wie die Cäsar-Chiffre, verwendet also verschobene
    lateinische Alphabete zur Verschlüsselung. Wie viele Alphabete und wie
    jedes von ihnen verschoben, wird durch den Schlüssel $k$ bestimmt, der in diesem
    Fall ein Wort ist. Es gibt also soviele geheime Alphabete, wie es Zeichen
    in $k$ gibt und jedes dieser Alphabete ist verschoben, so dass das $i$-te Alphabet
    mit dem $i$-ten Zeichen von $k$ beginnt.
    Alice wählt das Schlüsselwort $k = EDV$. Damit ergeben sich Klartext- und
    Geheimalphabete wie folgt:

    \begin{center}
        $\mathfrak{P}$: \texttt{A B C D E F G H I J K L M N O P Q R S T U V W X Y Z}\\
        $\mathfrak{S}_1$: \texttt{E F G H I J K L M N O P Q R S T U V W X Y Z A B C D}\\
        $\mathfrak{S}_2$: \texttt{D E F G H I J K L M N O P Q R S T U V W X Y Z A B C}\\
        $\mathfrak{S}_3$: \texttt{V W X Y Z A B C D E F G H I J K L M N O P Q R S T U}\\
    \end{center}

    Damit kann Alice den Klartext $M =$ \texttt{HALLOHALLO} verschlüsseln, indem Sie
    die Buchstaben des Klartexts wie bei der Cäsar Chiffre zuordnet und dabei die
    drei Geheimtextalphabete reihum verwendet. Da der Klartext mehr Zeichen
    enthält, als es Geheimtextalphabete gibt, fängt sie nach jeweils drei Zeichen
    wieder beim ersten Geheimtextalphabet an. Somit ergibt sich der Geheimtext
    $C =$ \texttt{LDGPRCEOGS}.
\end{example}

\begin{defi}{Moderne Verschlüsselungsverfahren}
    Moderne Verschlüsselungsverfahren werden, im Gegensatz zu klassischen Verfahren,
    nicht mehr auf Zeichen einer natürlichen Sprache angewandt. Stattdessen werden
    Zahlen bzw. Bits als Grundlage der Operationen verwendet. Damit können
    beliebige Daten verschlüsselt werden, nicht mehr nur Text. Damit sind moderne
    Verschlüsselungsverfahren den Anforderungen der heutigen digitalen Gesellschaft
    gewachsen.

    \textbf{Symmetrische Verfahren}:
    \begin{itemize}
        \item denselben Schlüssel zur Ver- und Entschlüsselung
        \item Problem des Schlüsselaustauschs
        \item Vertraulichkeit, aber keine zur Sicherstellung von Authentifizierung und Integrität
        \item[$\to$] Blockchiffren
            \subitem -  zu verschlüsselnde Daten in $m$ Blöcke derselben Größe aufgeteilt
            \subitem -  einfachster Fall: jeder der zuvor erzeugten Blöcke $m_i$ mit Schlüssel $k$
            \subitem zu verschlüsseltem Block $c_i$ derselben Größe übersetzt
        \item[$\to$] Stromchiffren
            \subitem -  Klartext wird bitweise anhand eines Schlüsselstroms verschlüsselt
            \subitem - Schlüsselstrom und Geheimtext haben dieselbe Länge wie der Klartext
    \end{itemize}

    \textbf{Asymmetrische Verfahren}:
    \begin{itemize}
        \item Die meisten asymmetrischen Verfahren basieren auf mathematischen Problemen, die nicht effizient zu lösen sind.
        \item
    \end{itemize}
\end{defi}

\begin{algo}{RSA Verschlüsselung}
    RSA basiert auf der Faktorisierung ganzes Zahlen, für die es keinen bekannten
    effizienten Algorithmus gibt. Dabei geht es darum, eine Zahl in ihre Primfaktoren
    zu zerlegen.

    Da beide Schlüssel des Schlüsselpaars zusammen funktionieren sollen, müssen sie
    nach einer festen Vorschrift erzeugt werden:
    \begin{enumerate}
        \item Wähle zwei Primzahlen $p, q$ mit $p \neq q$
        \item Berechne $N = p \cdot q$
        \item Berechne $\varphi(N) = (p-1)\cdot (q-1)$
        \item Wähle ein $e$, das teilerfremd zu $\varphi(N)$ ist mit $1 < e < \varphi(N)$
        \item Berechne $d$, sodass $e \cdot d \equiv 1 \mod \varphi(N)$
    \end{enumerate}

    Nach dieser Prozedur ist $(e,N)$ der \emph{public key} und $(d,N)$ der \emph{private Key}. Damit
    kann nun jeder Nachricht $M \in \N, 1 < M < N$ anhand folgender Formel verschlüsselt
    werden:
    $$
        C = M^e \mod N
    $$

    Die Entschlüsselung der verschlüsselten Nachricht $C$ kann dann anhand einer ähnlichen Formel durchgeführt werden:
    $$
        M = C^d \mod N
    $$
\end{algo}

\begin{example}{RSA Verschlüsselung}
    Damit Alice Bob eine Nachricht schreiben kann, muss Bob zunächst seinen \emph{public Key} zur Vefügung stellen.

    Zur Erzeugung seines Schlüsselpaars wählt Bob die Primzahlen $p=13$ und $q=17$.
    Damit ergibt sich $N = 221$ und $\varphi(N) = 192$.
    Anschließend wählt Bob $e=5$ und berechnet damit $d=77$.
    Damit kann Bob $(5, 221)$ als seinen \emph{public Key} an Alice geben und $(77, 221)$ behält er als \emph{private Key} für sich.

    Alice hat nun Bobs \emph{public Key} und möchte damit die Nachricht $M = 42$ verschlüsseln.
    Dazu berechnet Sie $C = 42^5 \mod 221 = 9$ und schickt die verschlüsselte Nachricht anschließend an Bob.

    Bob kann nun mit seinem \emph{private Key} die erhaltene Nachricht entschlüsseln und erhält $M = 9^77 \mod 221 = 42$.
\end{example}

\begin{defi}{Digitale Signatur}
    Um auch \emph{Authentifizierung}
    und \emph{Integritat} herstellen zu können, muss eine digitale Signatur verwendet
    werden. Dazu braucht es zunachst eine \emph{Hashfunktion}, die eine Art Fingerabdruck
    der Nachricht erzeugt.

    Nachdem mit der Hashfunktion der Hashwert der Nachricht berechnet wurde, wird
    dieser mit dem \emph{private Key} des Senders verschlüsselt.

    Danach berechnet
    der Empfänger selbst den Hashwert der empfangen Nachricht. Stimmen der
    selbst berechnete Hashwert und der entschlüsselte Hashwert überein ist die \emph{Integrität} der Nachricht sichergestellt. Da zudem das Entschlüsseln des Hashwertes mit
    dem öffentlichen Schlüssel des Senders funktioniert hat, ist der Sender authentifiziert,
    da dessen privater Schlüssel zur Erstellung der digitalen Signatur verwendet
    wurde.

    Es ist also notwendig, dass sich Sender und Empfanger vorab auf eine
    Hashfunktion einigen. Zusatzlich ist wichtig zu beachten, dass eine Digitale Signatur
    \emph{keine Vertraulichkeit} herstellt, dazu muss die Nachricht zusatzlich verschlüsselt
    werden.
\end{defi}

\begin{defi}{Hybride Chiffren}
    Die Kombination von symmetrischen und asymmetrischen
    Chiffren nennt man \emph{hybride Chiffren}.

    \begin{itemize}
        \item verhältnismäßig kurzer Schlüssel einer Blockchiffre wird asymmetrisch verschlüsselt und zum Empfänger einer Nachricht transportiert
        \item Kommunikation selbst läuft symmetrisch verschlüsselt ab
        \item Kombination erzielt guten Kompromiss zwischen Sicherheit und Rechenaufwand
    \end{itemize}
\end{defi}
