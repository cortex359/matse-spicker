\section{Codierung}
\subsection{Stellenwertsysteme}

\begin{defi}{Stellenwertsystem}
    Allgemein lässt sich der Wert einer Zahl in einem Stellenwertsystem zur Basis $B$ wie folgt ausdrücken ($d_i$ ist der Wert der $i$-ten Stelle, $r$ der Exponent der höchtswertigen Stelle):
    $$
        n_b = \sum^r_{i=0} B^i \cdot d_i
    $$
\end{defi}

\begin{algo}{Dezimal $\to$ Binär, Hexadezimal, $\ldots$}
    Sei $B = 2$ für Umrechnung ins Binärsystem, bzw. $B =16$ für das Hexadezimalsystem.

    Es gilt für eine umzurechnende Zahl $z$:

    $$
        \begin{aligned}
            z : B       & = z_0 \quad     &  & \text{Rest } r_0     \\
            z_0 : B     & = z_1 \quad     &  & \text{Rest } r_1     \\
            z_1 : B     & = z_2 \quad     &  & \text{Rest } r_2     \\
                        & \ldots          &  &                      \\
            z_{n-2} : B & = z_{n-1} \quad &  & \text{Rest } r_{n-1} \\
            z_{n-1} : B & = 0 \quad       &  & \text{Rest } r_n
        \end{aligned}
    $$

    Damit gilt dann: $(z)_{10} = (r_nr_{n-1}\ldots r_2r_1r_0)_B$ (also gelesen von \emph{unten nach oben}).\qed
\end{algo}

\begin{algo}{Binär $\to$ Hexadezimal}
    Sei eine Binärzahl $b$ gegeben mit $n \in 4\N$ Bits.

    Dann kann $b$ wie folgt in eine Hexadezimalzahl $h$ mit $m = \frac{n}{4}$ Zeichen umgeformt werden:

    $$
        \underbrace{b_{n-1}b_{n-2}b_{n-3}b_{n-4}}_{h_{m-1}} ~ \ldots ~ \underbrace{b_7b_6b_5b_4}_{h_1} ~ \underbrace{b_3b_2b_1b_0}_{h_0}
    $$

    Erinnerung: 4 Bits können binär nur 16 mögliche Werte annehmen!
\end{algo}

\newpage
\subsection{Zahlendarstellungen}

\begin{defi}{Einerkomplement}
    Das \emph{Einerkomplement} einer Binärzahl wird gebildet, indem man alle Bits negiert.

    Das erste Bit gibt dabei das Vorzeichen an.

    Nachteile: Doppelte Darstellung der Null, Subtraktion lässt sich nicht auf Addition mit einer negativen Zahl zurückführen
\end{defi}

\begin{defi}{Zweierkomplement}
    Das \emph{Zweierkomplement} einer Binärzahl wird gebildet, indem man das Einerkomplement bildet und zusätzlich $1$ addiert.

    Das erste Bit gibt dabei das Vorzeichen an.

    Vorteile: Subtraktion entspricht Addition mit einer negativen Zahl, keine doppelte Null
\end{defi}

\begin{defi}{Binäre Festkommazahlen}
    Die Umwandlung des ganzzahligen Anteils einer (dezimalen) Festkommazahl erfolgt analog zu ganzen Zahlen.

    Zusätzlich werden aber die Nachkommastellen mit entsprechend negativen Exponenten weiter \glqq verrechnet\grqq.
\end{defi}

\begin{defi}{Gleitkommazahlen nach \emph{IEEE 754}}
    Die Darstellung einer Gleitkommazahl
    $$
        x = s \cdot m \cdot b^e
    $$
    besteht aus
    \begin{itemize}
        \item Vorzeichen $s$ (1 Bit)
        \item Mantisse $m$ ($p$ Bits, $p_{\text{float}} = 23$, $p_{\text{double}} = 52$)
        \item Basis $b$ (bei normalisierten Gleitkommazahlen nach IEEE ist $b = 2$)
        \item Exponent $e$ ($r$ Bits, $r_{\text{float}} = 8$, $r_{\text{double}} = 11$)
    \end{itemize}
\end{defi}

\begin{algo}{Umrechnung in \emph{IEEE 754}}
    \begin{itemize}
        \item Umwandeln einer Dezimalzahl in eine \emph{binäre Festkommazahl} ohne Vorzeichen.
        \item Normalisieren der \emph{Mantisse}:\\
              Das Komma wird $n$ Stellen nach links verschoben, so dass dort nur noch eine 1 als ganzzahliger Anteil vorhanden ist. ($n$ ist bei Verschieben nach rechts negativ!)\\
              $M$ ist dann der \emph{Nachkommateil}.
        \item Bestimmen des \emph{Exponenten}: $E = (\text{bias } + n)_2$ (bias ist meist $127$, wenn nicht anders gegeben)
        \item Bestimmen des \emph{Vorzeichenbits} $s$
        \item Zusammensetzen der Gleitkommazahl
              $$
                  s \mid E \mid M
              $$
    \end{itemize}
\end{algo}

\subsection{Zeichendarstellungen}

\begin{defi}{UTF-8 Codierung}
    \begin{tabular}{| c | r | c |}
        \hline
        Unicode-Bereich       & \multicolumn{1}{|c|}{UTF-8-Codierung}   & Möglichkeiten        \\
        \hline
        0000 0000 - 0000 007F & 0xxx xxxx                               & 128 (7 Bits)         \\
        0000 0080 - 0000 07FF & 110x xxxx 10xx xxxx                     & 2.048 (11 Bits)      \\
        0000 0800 - 0000 FFFF & 1110 xxxx 10xx xxxx 10xx xxxx           & 65.536  (16 Bits)    \\
        0001 0000 - 0010 FFFF & 1111 0xxx 10xx xxxx 10xx xxxx 10xx xxxx & 2.097.152  (21 Bits) \\
        \hline
    \end{tabular}
\end{defi}

\subsection{Spezielle Codierungen}

\begin{algo}{Huffman-Codierung}
    \begin{enumerate}
        \item Schreibe Buchstaben mit Auftrittshäufigkeiten als \glqq Wald\grqq.
        \item Fasse die beiden Bäume mit der geringsten Auftrittshäufigkeiten zu einem neuen Baum zusammen, dabei werden die Auftrittshäufigkeiten addiert.
        \item Wiederhole, bis nur noch ein Baum existiert.
        \item Codierung eines Buchstaben entspricht dann dem \emph{Pfad} zum entsprechenden Blatt mit \glqq links\grqq $\to$ 0, \glqq rechts\grqq $\to$ 1
    \end{enumerate}

    Mittlere Codelänge:
    $$
        L(C) = \frac{\#\text{Bits der verschlüsselten Nachricht}}{\#{\text{Zeichen der verschlüsselten Nachricht}}}
    $$
\end{algo}

\begin{defi}{Hamming-Codierung}
    Bei der Hamming-Codierung werden Paritätsinformationen zu Daten hinzugefügt, um so mögliche Übertragungsfehler zu erkennen.

    Hamming-Codewörter haben die Länge $N = 2^k-1$, wobei $k$ Paritätsbits enthalten sind.
    Die Bits werden der Einfachheit halber bei Eins beginnend durchnummeriert.
    Die Paritätsbits stehen an den Stellen, deren Index eine 2er-Potenz ist.

    Sind $p_1, p_2, \ldots, p_k$ Paritätsbits, $d_1, d_2, \ldots, d_{N-k}$ Bits des Datenwortes und $c_1, c_2, \ldots, c_N$ die Bits des zu bildenden Codewortes, hat ein Codewort des so konstruierten Hamming-Codes die folgende Form:

    \begin{tabular}{| c | c | c || c | c | c | c || c | c | c | c | c | c | c | c || c | c | c |}
        \hline
        $c_1$ & $c_2$ & $c_3$ & $c_4$ & $c_5$ & $c_6$ & $c_7$ & $c_8$ & $c_9$ & $c_{10}$ & $c_{11}$ & $c_{12}$ & $c_{13}$ & $c_{14}$ & $c_{15}$ & $c_{16}$ & $c_{17}$ & \ldots \\
        \hline
        $p_0$ & $p_1$ & $d_1$ & $p_2$ & $d_2$ & $d_3$ & $d_4$ & $p_3$ & $d_5$ & $d_6$    & $d_7$    & $d_8$    & $d_9$    & $d_{10}$ & $d_{11}$ & $p_4$    & $d_{12}$ & \ldots \\
        \hline
    \end{tabular}\\

    Dabei wird jedem Paritätsbit eine spezielle Bitmaske zugewiesen:

    \begin{tabular}{| c || c | c | c || c | c | c | c || c | c | c | c | c | c | c | c || c |}
        \hline
                       & $c_1$ & $c_2$ & $c_3$ & $c_4$ & $c_5$ & $c_6$ & $c_7$ & $c_8$ & $c_9$ & $c_{10}$ & $c_{11}$ & $c_{12}$ & $c_{13}$ & $c_{14}$ & $c_{15}$ & \ldots \\
        \hline
        Bitmaske $p_0$ & $p_0$ & $p_1$ & 1     & $p_2$ & 1     & 0     & 1     & $p_3$ & 1     & 0        & 1        & 0        & 1        & 0        & 1        & \ldots \\
        Bitmaske $p_1$ & $p_0$ & $p_1$ & 1     & $p_2$ & 0     & 1     & 1     & $p_3$ & 0     & 1        & 1        & 0        & 0        & 1        & 1        & \ldots \\
        Bitmaske $p_2$ & $p_0$ & $p_1$ & 0     & $p_2$ & 1     & 1     & 1     & $p_3$ & 0     & 0        & 0        & 1        & 1        & 1        & 1        & \ldots \\
        Bitmaske $p_3$ & $p_0$ & $p_1$ & 0     & $p_2$ & 0     & 0     & 0     & $p_3$ & 1     & 1        & 1        & 1        & 1        & 1        & 1        & \ldots \\
        \hline
    \end{tabular}\\

    Damit gilt:
    $$
        \begin{aligned}
            c_1 = p_0 & = c_3 \oplus c_5 \oplus c_7 \oplus c_9 \oplus c_{11} \oplus c_{13} \oplus c_{15} \oplus \ldots    \\
                      & \iff \text{jedes ungerade Datenbit}                                                               \\
            c_2 = p_1 & = c_3 \oplus c_6 \oplus c_7 \oplus c_{10} \oplus c_{11} \oplus c_{14} \oplus c_{15} \oplus \ldots \\
                      & \iff \text{ein Datenbit rechts von $p_1$, zwei überspringen, zwei einberechnen, $\ldots$}         \\
            c_4 = p_2 & = c_5 \oplus c_6 \oplus c_7 \oplus c_{12} \oplus c_{13} \oplus c_{14} \oplus c_{15} \oplus \ldots \\
                      & \iff \text{drei Datenbit rechts von $p_2$, vier überspringen, vier einberechnen, $\ldots$}
        \end{aligned}
    $$
\end{defi}

\begin{algo}{Fehlererkennung beim Hamming-Code}
    Situation: Empfangen eines Hamming-codierten Datensatzes

    \begin{enumerate}
        \item Erneutes Berechnen der \emph{Paritätsbits}
        \item Erkennen, welche neu berechneten Paritätsbits $p'_i$ \emph{verschieden} sind zu den empfangenen Paritätsbits $p_i$
        \item Bitfehler ist in dem Datenbit passiert, in dem gilt:
              \subitem $p_i = p'_i \implies$ Bitmaske für $p_i$ ist $0$
              \subitem $p_i \neq p'_i \implies$ Bitmaske für $p_i$ ist $1$
    \end{enumerate}

    Es wird angenommen, dass nur ein Bitfehler in den Datenbits passiert ist.

    Anmerkung: Ist lediglich ein Paritätsbit verschieden, dann ist ein Übertragungsfehler in dem betreffenden Paritätsbit selbst aufgetreten, da alle Datenbits zur Berechnung von mindestens zwei Paritätsbits verwendet werden.
\end{algo}
