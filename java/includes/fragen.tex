Java:
 - Java Compiler
 - Java Laufzeitumgebung / Java Virtual Machine / Java Bytecode-Interpreter

Casting
Wrapper Classes, Autoboxing

Klassen, Vererbung
    überschreiben
    überladen
    abstrakte Methoden (Signatur ohne Implementierung)
    statische Methoden

\begin{defi}{}

\end{defi}


Bindungsarten
 - Statischer und Dynamischer Typ

rekursive Algorithmen

checked und unchecked exceptions
  - Checked Exceptions müssen mit try-catch gefangen werden
  - Unchecked Exceptions können gefangen werden, müssen aber nicht

Generics

    public class SimpleBox<T>
Dabei nennt man T den generischen Typ von SimpleBox. Allgemein spricht man von Typvariablen oder Generics.


Lambdas

try {
    FileReader f = new FileReader("daten.txt");
    kopf =  f.readline();
} catch (IOException e) {
    System.out.println("Error: " + e);
}




\section{Fragen}
a) Das Programm, mit dem der Java-Quellcode übersetzt wird, nennt man:
Java-Compiler
b) Das Programm, mit dem die Java-Klassen ausgeführt werden, nennt man:
Java Laufzeitumgebung oder Java Virtual Machine oder Java Bytecode-Interpreter
c) Eine Ausnahme, die ein Entwickler im Code in jedem Fall behandeln muss, nennt man:
Checked Exception
d) Die Definition einer Methode, die in einer Oberklasse bereits definiert ist, nennt man in Java:
überschreiben
e) Die Definition mehrerer Methoden mit gleichem Namen und unterschiedlicher Signatur nennt man:
überladen
f) Eine Methode, bei der ausschließlich die Signatur definiert und keine Implementierung angegeben ist, nennt man:
abstrakt (Ein Interface darf nur abstrakte Methoden haben, daher kann mandas Schluesselwort abstract hier auch weglassen).
g) Eine Methode, die ohne eine Objektinstanz aufgerufen werden kann, nennt man:
statisch
h) Welche Art von Klasse wird im folgenden Code-Ausschnitt erzeugt?
Funktion f = new Funktion () {
public double getY( double x) {
return x*x;
}
};
-> anonyme innere Klasse
i) Zu welcher Art von Klassen gehören die Klassen Integer, Float und Character?
Wrapper-Klassen
j) Sie wollen einen String aus vielen einzelnen Strings zusammenbauen. Welche Klasse bietet sich aus Geschwindigkeits-Gründen an?
StringBuilder (oder StringBuffer)
k) Die Anweisung, um eine Schleife aus dem Schleifeninneren heraus abzubrechen, heisst:
break
l) Die Variable Bruch b; kann nicht nur Objekte der Klasse Bruch aufnehmen. Welche Objekte kann b noch aufnehmen?
Objekte einer Unterklasse von Bruch
m) Welchen Konstruktor fügt Java einer Klasse automatisch zu, wenn kein Konstruktor implementiert wurde?
Den Default-Konstruktor (oder den parameterlosen Konstruktor)
n) Wie nennt man den Teil in spitzen Klammern im folgenden Ausdruck?
ArrayList<String> x = new ArrayList<String>();
-> Typparameter (Generic)
o) Schreiben Sie den Java-Code auf, den Sie benötigen, um eine Datei daten.txt zu öffnen und die erste Zeile der Datei in den String kopf einzulesen. Berücksichtigen Sie auch Exceptions.

String kopf = null;
try {
    Scanner sc = new Scanner(new File("daten.txt"));
    kopf = sc.nextLine();
} catch (FileNotFoundException e) {
    // ...
}

p) Erläutern Sie anhand der folgenden Programmzeilen kurz die Begriffe
(1) Interface
(2) Anonyme innere Klasse
 Function f = new Function () {
 public double getY( double x) {
 return x*x;
 }
 };

(1) Interfaces (hier "Function"): Interfaces enthalten nur die Koepfe von
    Methoden. Die Unterklassen muessen diese Methoden implementieren.
(2) Anonyme innere Klassen (hier eine Unterklasse von `Function'):
    Unterklassen ohne eigenen Namen, die in einer Methode deklariert
    werden.